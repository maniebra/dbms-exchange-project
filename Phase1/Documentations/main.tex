\documentclass{book}

\usepackage{amsmath}
\usepackage{amsfonts}
\usepackage{amssymb}
\usepackage{graphicx}
\usepackage[hidelinks]{hyperref}
\usepackage{fontspec}
\usepackage{xepersian}

\def\ojoin{\setbox0=\hbox{$\bowtie$}%
  \rule[-.02ex]{.25em}{.4pt}\llap{\rule[\ht0]{.25em}{.4pt}}}
\def\leftouterjoin{\mathbin{\ojoin\mkern-5.8mu\bowtie}}
\def\rightouterjoin{\mathbin{\bowtie\mkern-5.8mu\ojoin}}
\def\fullouterjoin{\mathbin{\ojoin\mkern-5.8mu\bowtie\mkern-5.8mu\ojoin}}

\makeatletter
\renewcommand\thesection{\@arabic\c@section.\@arabic\c@chapter}
\makeatother
\makeatletter
\renewcommand\thesubsection{\@arabic\c@subsection.\@arabic\c@section.\@arabic\c@chapter}
\makeatother

\newcommand{\subtitle}[1]{%
  \posttitle{%
    \par\large\textbf{#1}\vskip0.5em}%
}

\settextfont{x-nima.ttf}

\title{
    { \includegraphics[width=0.33\linewidth]{Logo BOW.png} \\
        \normalsize{دانشگاه صنعتی شریف}} \\ \vspace{1cm}
    پروژه‌ی صرافی ارز دیجیتال: پایگاه داده - فاز اول \\ \vspace{0.5cm} \large{استاد: دکتر مهدی آخی} \\
    \vspace{1cm} {شمارۀ تیم: 14}}
\date{\vspace{1cm} بهار 1403}
\author{
  مانی ابراهیمی\\
  {401170491}
  \and
  محمدامین حیدری\\
  {401170553}
  \and
  محمد جعفری‌پور\\
  {401105797}
}

\begin{document}
    \maketitle
    \newpage
    
    \tableofcontents
    \newpage

        \chapter{کلیت فاز اول پروژه}

        \section{شرح}
        در این فاز تلاش شده تا یک پایگاه داده‌ی مرتبط با یک صرافی ارز دیجیتال طراحی شود. این پایگاه داده شامل موجودیت‌هایی مانند کاربر و کیف پول و تراکنش است. همچنین برای هر موجودیت روابطی با موجودیت‌های دیگر نیز تعریف شده است. در ادامه به توضیح هر یک از موجودیت‌ها و روابط آن‌ها با موجودیت‌های دیگر پرداخته‌ایم. همچنین در انتها پاسخ به 10 پرسش جبر رابطه‌ای داده شده نیز آمده است. مخزن یا همان \lr{repository} این پروژه در \href{https://github.com/maniebra/dbms-exchange-project}{\underline{اینجا}}\footnote{در صورتی که لینک برای شما کار نمی‌کند، از آدرس \href{https://github.com/maniebra/dbms-exchange-project} استفاده نمایید.} قابل مشاهده است.

        \section{تقسیم وظایف}
        تیم این پروژه متشکل از سه نفر بود که برای سادگی در سند تقسیم وظایف، برای آن‌ها از اسم کوتاه استفاده کردیم:
        
        \begin{table}[h]

        \centering
        \begin{tabular}{|c|c|c|}
            \hline
            نام کوتاه & نام کامل & شماره دانشجویی \\\hline
            \lr{Mani} & مانی ابراهیمی & 401170491 \\\hline
            \lr{Mamadamin} & محمدامین حیدری & 401170553 \\\hline
            \lr{Mamal} & محمد جعفری‌پور & 401105797 \\\hline
        \end{tabular}
        \caption{جدول اعضای تیم در جدول تقسیم وظایف}
        \end{table}

        جدول تقسیم وظایف نیز از \hyperlink{https://docs.google.com/spreadsheets/d/1x1Guh4HTWLyG9GTomZEtesp5cjIGez9m9Day3bS_kgM/edit?usp=sharing}{\underline{اینجا}}\footnote{در صورتی که این لینک برای شما کار نمی‌کند، میتوانید از آدرس \\\scriptsize\lr{https://docs.google.com/spreadsheets/d/1x1Guh4HTWLyG9GTomZEtesp5cjIGez9m9Day3bS\_kgM/edit?usp=sharing}\small\\ استفاده نمایید.} قابل مشاهده است.
        \chapter{دیاگرام های \lr{ER}}
        \section{شرح}
        در این بخش تلاش بر این بود که کلیت پایگاه داده‌ی مورد نظر را با استفاده از دیاگرام‌های \lr{ER} نمایش دهیم. ابتدا دیاگرام \lr{ER} اصلی را نمایش داده‌ایم و سپس به تفکیک بخش‌های مختلف آن پرداخته‌ایم.
        \section{توضیح هر موجودیت}
        در ادامه، برای هر موجودیت حاضر در این دیاگرام توضیحی آمده:

        \subsection{\lr{User}}
        موجودیت کاربر یا همان \lr{user}، که دارای صفات گفته شده از جمله نام و نام خانوادگی و شناسه ملی و شماره تماس و ایمیل و رمز عبور و سایر موارد است. این موجودیت برای کاربران اصلی‌ترین موجودیت بوده چرا که اطلاعات خود هر کاربر را در این موجودیت ذخیره می‌کنیم. همچنین به یک موجودیت کیف پول متصل است که باعث می‌شود هر کاربر یک کیف پول داشته باشد.

        \subsection{\lr{Wallet}}
        موجودیت کیف پول یا همان \lr{wallet}، که دارای صفات گفته شده از جمله موجودیت کاربر و موجودی و ارز و سایر موارد است. این موجودیت برای ذخیره‌ی اطلاعات مربوط به کیف پول هر کاربر استفاده می‌شود. همچنین به یک موجودیت تراکنش متصل است که باعث می‌شود هر کیف پول دارای تراکنش باشد.

        \subsection{\lr{Transactions}}
        موجودیت تراکنش یا همان \lr{transactions}، که دارای صفات گفته شده از جمله موجودیت کیف پول و نوع تراکنش و مبلغ و تاریخ و سایر موارد است. این موجودیت برای ذخیره‌ی اطلاعات مربوط به تراکنش‌های هر کیف پول استفاده می‌شود.


        \chapter{سوالات جبر رابطه‌ای}
        \section{شرح}
        در این بخش پاسخ به 10 سوال جبر رابطه‌ای\footnote{\lr{Relational Algebra}} آمده است.
        \section{پاسخ به سوالات}
        \subsection{سوال 1}


        \subsection{سوال 2}


        \subsection{سوال 3}


        \subsection{سوال 4}


        \subsection{سوال 5}


        \subsection{سوال 6}


        \subsection{سوال 7}


        \subsection{سوال 8}


        \subsection{سوال 9}


        \subsection{سوال 10}

\end{document}